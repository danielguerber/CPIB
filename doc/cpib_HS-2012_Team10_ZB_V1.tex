\documentclass[a4paper,10pt]{article}

\usepackage{graphicx} 

\usepackage[T1]{fontenc}

\usepackage{fancyhdr}

%Metadata
\title{CPIB Zwischenbericht}
\author{Daniel G\"urber}

\begin{document}

\part*{Zwischenbericht Daniel G\"urber)}
\section{Abstract}
Dieses Dokument zeigt auf wie die geplanten Erweiterungen aussehen und
wie sie in der Syntax umgesetzt werden. Die geplanten Erweiterungen
sind verkettete Vergleichsoperatoren und Prozeduraufrufe ohne call.
\section{Erweiterung}
\subsection{Verkettete Vergleichsoperatoren}
\subsubsection{Ziel}
Das Ziel ist es, IML so zu erweitern, dass Vergleichsoperatoren verkettet verwendet werden k\"onnen. Ausdr\"ucke sollen nicht mehrmals
geschrieben werden m\"ussen, um sie mit mehreren Werten zu vergleichen.
\subsection{Beispiele}
\begin{itemize}
\item x < y < z anstatt x < y cand y < z
\item x = y = z anstatt x = y cand y = z
\item a > b > c >= d anstatt a > b cand b > c cand c >= d
\end{itemize}
\subsection{Aufrufe ohne call}
\subsubsection{Ziel}
Das Ziel ist es, IML so zu erweitern, dass Prozeduraufrufe ohne das
Schlüsselwort call m\"oglich sind.
\subsection{Beispiele}
\begin{itemize}
\item doSomething(); anstatt call doSomething;
\item calculateIt(x init); anstatt call calculateIt(x init);
\end{itemize}
\section{Lexikalische Syntax}
Die lexikalische Syntax \"andert sich im Vergleich zu der von IML v3
nur dahingehend, dass das Schl\"usselwort call gestrichen wird.
\section{Grammatikalische Syntax}
Die grammatikalische Syntax entspricht gr\"osstenteils dervon IML v3 mit eingebauten Hilfskonstrukten, die folgenden Definitionen werden durch die Erweiterungen ge\"andert:

\begin{tabbing}
term1 ::= term2 repTerm2
\end{tabbing}
\begin{tabbing}
repTerm2 ::\= = \= RELOPR term2 repTerm2\\
\> |\> epsilon
\end{tabbing}
\begin{tabbing}
cmd ::\= = \= SKIP\\
\> |\> expr auxExprCmd\\
\> |\> IF LPAREN expr RPAREN blockCmd ELSE blockCmd\\
\> |\> WHILE LPAREN expr RPAREN blockCmd\\
\> |\> QUESTMARK expr\\
\> |\> EXCLAMARK expr\\
\end{tabbing}
\begin{tabbing}
auxExprCmd ::\= = \= BECOMES expr\\
\> |\> auxGlobInitList\\
\end{tabbing}
\begin{tabbing}
auxGlobInitList ::\= = \= INIT LPAREN globInitList RPAREN\\
\> |\> epsilon\\
\end{tabbing}
Weitere Details k\"onnen dem angeh\"angten File Grammar\_IML.sml entnommen werden. 
\section{Kontext- und Typeinschr\"ankungen}
Bei den Vergleichsoperatoren dürfen weiterhin nur Typen verglichen werden, welche auch Vergleichbar miteinander sind. Der Kontext wird so
eingeschränkt, dass verkettete Vergleichsoperatoren die Richtung nicht \"andern d\"urfen, das heisst wenn ein < oder <= Operator verwendet wurde, darf also kein > oder >= Operator verwendet werden und umgekehrt.\\
Das Weglassen des call Schl\"usselwortes führt dazu, dass
sichergestellt werden muss, dass nur Prozeduraufrufe ohne 
Zuweisungoperator dastehen k\"onnen und das Prozeduraufrufe nicht in
Expressions verwendet werden d\"urfen. Zudem darf die globale
Init-Liste nicht hinter Funktionsaufrufen verwendet werden. 
\section{Andere Programmiersprachen}
Vergleichsoperatoren lassen sich in den meisten anderen Programmiersprachen grunds\"atzlich nicht verketten; Grund daf\"ur ist, dass, wenn Expressions und Commands nicht sauber getrennt sind, der Command zwischen zwei Vergleichs-operatoren zweimal ausgef\"uhrt wird.\\
Prozeduraufrufe ohne call sind in vielen anderen Programmiersprachen
m\"oglich zum Beispiel in den von C abgeleiteten Sprachen.
\section{Begr\"undungen Entwurf}
\subsection{Vergleichsoperaoren}
Die Einschr\"ankung auf eine Richtung bei den Vergleichsoperatoren
f\"uhrt zu verbesserter Leserlichkeit, da die Werte von links nach 
rechts immer gr\"osser beziehungsweise kleiner werden.
\subsection{Prozeduraufruf}
F\"ur das Weglassen des call Schl"usselwortes muss man bei der 
globalen Init-Liste Klammern hinzuf\"ugen, bei anderen Umsetzungen 
entstehen folgende Probleme:\\
\begin{itemize}
\item Werden die Klammern weggelassen, kann der Parser nicht zwischen
globInitList und exprList unterscheiden, da beides kommagetrennte
Listen sind.
\item Wird globInitList als Teil des auxExprCmd aufgefasst entsteht
ein Konflikt mit factor, da an beiden Orten optional ein INIT
verwendet wird.
\end{itemize}

\end{document}

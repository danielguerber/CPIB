\documentclass[a4paper,10pt]{article}

\usepackage{graphicx} 

\usepackage[T1]{fontenc}

%Metadata
\title{CPIB Zwischenbericht}
\author{Daniel G\"urber}

\begin{document}
\maketitle
\newpage
\tableofcontents
\newpage

\section{Abstract}
\section{Erweiterung}
\subsection{Verkettete Vergleichsoperatoren}
\subsubsection{Ziel}
Das Ziel ist es, IML so zu erweitern, dass Vergleichsoperatoren verkettet verwendet werden k\"onnen. Ausdr\"ucke sollen nicht mehrmals
geschrieben werden m\"ussen, um sie mit mehreren Werten zu vergleichen. Es soll nicht m\"oglich sein, die Richtung zu wechseln; wenn ein < oder <= Operator verwendet wurde, darf also kein > oder >=
Operator verwendet werden und umgekehrt.
\subsection{Beispiele}
\begin{itemize}
\item x < y < z anstatt x < y cand y < z
\item x = y = z anstatt x = y cand y = z
\item x > y = z anstatt x > y cand y = z
\item a > b > c >= d anstatt a > b cand b > c cand c >= d
\end{itemize}
\subsection{Grammatik\"anderungen}
Um dies umzusetzen wird die Grammatik von IML angepasst:\\
Die Definition von term1 wird von
\begin{tabbing}
term1 ::= term2 [RELOPR term2]
\end{tabbing}
zu
\begin{tabbing}
term1 ::= term2 \{RELOPR term2\}
\end{tabbing}
ge\"andert.
\subsection{Aufrufe ohne call}
\subsubsection{Ziel}
Das Ziel ist es, IML so zu erweitern, dass Prozeduraufrufe ohne das
Schlüsselwort call möglich sind.
\subsection{Beispiele}
\begin{itemize}
\item doSomething(); anstatt call doSomething;
\item calculateIt(x init); anstatt call calculateIt(x init);
\end{itemize}
\subsection{Grammatik\"anderungen}
Um dies umzusetzen wird die Grammatik von IML angepasst:\\\\
Die Definition von cmd wird von
\begin{tabbing}
cmd ::\= = \= SKIP\\
\> |\> expr BECOMES expr\\
\> |\> IF LPAREN expr RPAREN blockCmd ELSE blockCmd\\
\> |\> WHILE LPAREN expr RPAREN blockCmd\\
\> |\> CALL IDENT exprList [INIT globInitList]\\
\> |\> QUESTMARK expr\\
\> |\> EXCLAMARK expr\\
\end{tabbing}
zu
\begin{tabbing}
cmd ::\= = \= SKIP\\
\> |\> expr [BECOMES expr]\\
\> |\> IF LPAREN expr RPAREN blockCmd ELSE blockCmd\\
\> |\> WHILE LPAREN expr RPAREN blockCmd\\
\> |\> QUESTMARK expr\\
\> |\> EXCLAMARK expr\\
\end{tabbing}
ge\"andert.\\\\
Die Definition von factor wird von
\begin{tabbing}
factor ::\= = \= LITERAL\\
\> |\> IDENT [INIT | exprList]\\
\> |\> monadicOpr factor\\
\> |\> LPAREN expr RPAREN\\
\end{tabbing}
zu
\begin{tabbing}
factor ::\= = \= LITERAL\\
\> |\> IDENT [INIT | exprList [INIT LPAREN globInitList RPAREN]\\
\> |\> monadicOpr factor\\
\> |\> LPAREN expr RPAREN\\
\end{tabbing}
ge\"andert.
\section{Lexikalische Syntax}
\section{Grammatikalische Syntax}
\section{Kontext- und Typeinschr\"ankungen}
\section{andere Programmiersprachen}
\section{Begr\"undungen Entwurf}

\end{document}